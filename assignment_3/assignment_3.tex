%%%%%%%%%%%%%%%%%%%%%%%%%%%%%%%%%%%%%%%%%
% Short Sectioned Assignment
% LaTeX Template
% Version 1.0 (5/5/12)
%
% This template has been downloaded from:
% http://www.LaTeXTemplates.com
%
% Original author:
% Frits Wenneker (http://www.howtotex.com)
%
% License:
% CC BY-NC-SA 3.0 (http://creativecommons.org/licenses/by-nc-sa/3.0/)
%
%%%%%%%%%%%%%%%%%%%%%%%%%%%%%%%%%%%%%%%%%

%----------------------------------------------------------------------------------------
%	PACKAGES AND OTHER DOCUMENT CONFIGURATIONS
%----------------------------------------------------------------------------------------

\documentclass[paper=a4, fontsize=11pt]{scrartcl} % A4 paper and 11pt font size

\usepackage[T1]{fontenc} % Use 8-bit encoding that has 256 glyphs
\usepackage{fourier} % Use the Adobe Utopia font for the document - comment this line to return to the LaTeX default
\usepackage[english]{babel} % English language/hyphenation
\usepackage{amsmath,amsfonts,amsthm} % Math packages

\usepackage{lipsum} % Used for inserting dummy 'Lorem ipsum' text into the template

\usepackage{sectsty} % Allows customizing section commands
\allsectionsfont{\centering \normalfont\scshape} % Make all sections centered, the default font and small caps

%%%%%%%%%%%%%%%%%%%%%%%%%%%%%%%%%%%%%%%%%%%%%%%%%%%%%%%%%%%%%%%%%%%%%%%%%%%%%%%%%%%%%%%
\usepackage{siunitx} % Provides the \SI{}{} and \si{} command for typesetting SI units
\usepackage{graphicx} % Required for the inclusion of images
\usepackage{amsmath} % Required for some math elements
\usepackage{enumerate} % Required for the enumerate function
\usepackage{circuitikz} % Required for the drawing of circuit diagrams
\usepackage{graphicx}
\usepackage{xfrac}
\usepackage{float}
%%%%%%%%%%%%%%%%%%%%%%%%%%%%%%%%%%%%%%%%%%%%%%%%%%%%%%%%%%%%%%%%%%%%%%%%%%%%%%%%%%%%%%%

\usepackage{fancyhdr} % Custom headers and footers
\pagestyle{fancyplain} % Makes all pages in the document conform to the custom headers and footers
\fancyhead{} % No page header - if you want one, create it in the same way as the footers below
\fancyfoot[L]{} % Empty left footer
\fancyfoot[C]{} % Empty center footer
\fancyfoot[R]{\thepage} % Page numbering for right footer
\renewcommand{\headrulewidth}{0pt} % Remove header underlines
\renewcommand{\footrulewidth}{0pt} % Remove footer underlines
\setlength{\headheight}{13.6pt} % Customize the height of the header

\numberwithin{equation}{section} % Number equations within sections (i.e. 1.1, 1.2, 2.1, 2.2 instead of 1, 2, 3, 4)
\numberwithin{figure}{section} % Number figures within sections (i.e. 1.1, 1.2, 2.1, 2.2 instead of 1, 2, 3, 4)
\numberwithin{table}{section} % Number tables within sections (i.e. 1.1, 1.2, 2.1, 2.2 instead of 1, 2, 3, 4)

\setlength\parindent{0pt} % Removes all indentation from paragraphs - comment this line for an assignment with lots of text

%----------------------------------------------------------------------------------------
%	TITLE SECTION
%----------------------------------------------------------------------------------------

\newcommand{\horrule}[1]{\rule{\linewidth}{#1}} % Create horizontal rule command with 1 argument of height

\title{	
\normalfont \normalsize 
\textsc{Charles Darwin University, School of Engineering} \\ [25pt] % Your university, school and/or department name(s)
\horrule{0.5pt} \\[0.4cm] % Thin top horizontal rule
\huge Electrical Machines and Power Systems Assignment 1 \\ % The assignment title
\horrule{2pt} \\[0.5cm] % Thick bottom horizontal rule
}

\author{Shane Reynolds} % Your name

\date{\normalsize\today} % Today's date or a custom date

\begin{document}

\maketitle % Print the title

%----------------------------------------------------------------------------------------
%	PROBLEM 1
%----------------------------------------------------------------------------------------

\section*{Assignment 1.1}

\begin{circuitikz}
	\draw (0,0)
	to [short, o-] (1,0)
	to [R, l^=1\si{\ohm}] (2,0)
	to [short, -o] (3,0)
	;
\end{circuitikz}

\begin{align*}
	R_{eq} = 1 \si{\ohm}
\end{align*}

\begin{circuitikz}
	\draw (0,0)
	to [short, o-] (1,0)
	to [R, l^=2\si{\ohm}] (2,0)
	to [short, -o] (3,0)
	;
\end{circuitikz}

\begin{align*}
R_{eq} = 2 \si{\ohm}
\end{align*}

\begin{circuitikz}
	\draw (0,0)
	to [short, o-] (1,0)
	to [R, l^=3\si{\ohm}] (2,0)
	to [short, -o] (3,0)
	;
\end{circuitikz}

\begin{align*}
R_{eq} = 3 \si{\ohm}
\end{align*}

\newpage

\begin{circuitikz}
	\draw (0,0)
	to [short, o-] (1,0)
	to [R, l^=1\si{\ohm}] (2,0)
	to [short] (3,0)
	to [R, l^=2\si{\ohm}] (4,0)
	to [short, -o] (5,0)
	;
\end{circuitikz}

\begin{align*}
R_{eq} &= R_{1} + R_{2} \\
&= 3 \si{\ohm} 
\end{align*}

\begin{circuitikz}
	\draw (0,0)
	to [short, o-] (1,0)
	to [R, l^=1\si{\ohm}] (2,0)
	to [short] (3,0)
	to [R, l^=3\si{\ohm}] (4,0)
	to [short, -o] (5,0)
	;
\end{circuitikz}

\begin{align*}
R_{eq} &= R_{1} + R_{3} \\
&= 4 \si{\ohm} 
\end{align*}

\begin{circuitikz}
	\draw (0,0)
	to [short, o-] (1,0)
	to [R, l^=2\si{\ohm}] (2,0)
	to [short] (3,0)
	to [R, l^=3\si{\ohm}] (4,0)
	to [short, -o] (5,0)
	;
\end{circuitikz}

\begin{align*}
R_{eq} &= R_{2} + R_{3} \\
&= 5 \si{\ohm} 
\end{align*}

\begin{circuitikz}
	\draw (0,0)
	to [short, o-] (1,0)
	to [R, l^=1\si{\ohm}] (2,0)
	to [short] (3,0)
	to [R, l^=2\si{\ohm}] (4,0)
	to [short] (5,0)
	to [R, l^=3\si{\ohm}] (6,0)
	to [short, -o] (7,0)
	;
\end{circuitikz}

\begin{align*}
R_{eq} &= R_{1} + R_{2} + R_{3} \\
&= 6 \si{\ohm} 
\end{align*}

\begin{circuitikz}
	\draw (0,0)
	to [short, o-] (1,0)
	to [R, l^=1\si{\ohm}] (1,-2)
	to [short, -o] (0,-2)
	;
	
	\draw (1,0)
	to [short] (3,0)
	to [R, l^=2\si{\ohm}] (3,-2)
	to [short] (1,-2)
	;
\end{circuitikz}

\begin{align*}
R_{eq} &= R_{1} || R_{2} \\
&= \frac{2}{3} \si{\ohm} 
\end{align*}

\begin{circuitikz}
	\draw (0,0)
	to [short, o-] (1,0)
	to [R, l^=1\si{\ohm}] (1,-2)
	to [short, -o] (0,-2)
	;
	
	\draw (1,0)
	to [short] (3,0)
	to [R, l^=3\si{\ohm}] (3,-2)
	to [short] (1,-2)
	;
\end{circuitikz}

\begin{align*}
R_{eq} &= R_{1} || R_{3} \\
&= \frac{3}{4} \si{\ohm} 
\end{align*}

\begin{circuitikz}
	\draw (0,0)
	to [short, o-] (1,0)
	to [R, l^=2\si{\ohm}] (1,-2)
	to [short, -o] (0,-2)
	;
	
	\draw (1,0)
	to [short] (3,0)
	to [R, l^=3\si{\ohm}] (3,-2)
	to [short] (1,-2)
	;
\end{circuitikz}

\begin{align*}
R_{eq} &= R_{2} || R_{3} \\
&= \frac{6}{5} \si{\ohm} 
\end{align*}

\begin{circuitikz}
	\draw (0,0)
	to [short, o-] (1,0)
	to [R, l^=1\si{\ohm}] (1,-2)
	to [short, -o] (0,-2)
	;
	
	\draw (1,0)
	to [short] (3,0)
	to [R, l^=2\si{\ohm}] (3,-2)
	to [short] (1,-2)
	;
	
	\draw (3,0)
	to [short] (5,0)
	to [R, l^=3\si{\ohm}] (5,-2)
	to [short] (3,-2)
	;
\end{circuitikz}

\begin{align*}
R_{eq} &= R_{1} || R_{2} || R_{3} \\
&= \frac{6}{11} \si{\ohm} 
\end{align*}

\begin{circuitikz}
	\draw (0,0)
	to [short, o-] (1,0)
	to [R, l^=1\si{\ohm}] (3,0)
	to [short] (4,0)
	to [R, l^=2\si{\ohm}] (4,-2)
	to [short, -o] (0,-2)
	;
	
	\draw (4,0)
	to [short] (6,0)
	to [R, l^=3\si{\ohm}] (6,-2)
	to [short] (4,-2)
	;	
\end{circuitikz}

\begin{align*}
R_{eq} &= R_{1} + R_{2} || R_{3} \\
&= \frac{11}{5} \si{\ohm} 
\end{align*}

\newpage

\begin{circuitikz}
	\draw (0,0)
	to [short, o-] (1,0)
	to [R, l^=2\si{\ohm}] (3,0)
	to [short] (4,0)
	to [R, l^=1\si{\ohm}] (4,-2)
	to [short, -o] (0,-2)
	;
	
	\draw (4,0)
	to [short] (6,0)
	to [R, l^=3\si{\ohm}] (6,-2)
	to [short] (4,-2)
	;
\end{circuitikz}

\begin{align*}
R_{eq} &= R_{2} + R_{1} || R_{3} \\
&= \frac{11}{4} \si{\ohm} 
\end{align*}

\begin{circuitikz}
	\draw (0,0)
	to [short, o-] (1,0)
	to [R, l^=3\si{\ohm}] (3,0)
	to [short] (4,0)
	to [R, l^=2\si{\ohm}] (4,-2)
	to [short, -o] (0,-2)
	;
	
	\draw (4,0)
	to [short] (6,0)
	to [R, l^=1\si{\ohm}] (6,-2)
	to [short] (4,-2)
	;
\end{circuitikz}

\begin{align*}
R_{eq} &= R_{3} + R_{2} || R_{1} \\
&= \frac{11}{3} \si{\ohm} 
\end{align*} \\

\newpage
%----------------------------------------------------------------------------------------
%	PROBLEM 2
%----------------------------------------------------------------------------------------

\section*{Assignment 1.2}

\begin{circuitikz}
	
	\draw (0,0) node[label={[font=\footnotesize]left:C}] {}
	to [battery1, l^=12 \si{\volt}] (0,4)
	to [short, -*] (0,4) node[label={[font=\footnotesize]left:A}] {}
	to [R, l^=1.6\si{\ohm}] (2,4)
	to [R, l^=4\si{\ohm}] (2,0)
	to [short, -*] (0,0)
	;
	
	\draw (2,4)
	to [short, -*] (4,4) node[label={[font=\footnotesize]right:B}] {}
	to [R, l^=6\si{\ohm}] (4, 0)
	to [short] (2,0)
	;
	
\end{circuitikz}

Node C \\
$v_{C} = 0 \si{\volt}$, since this is the ground node.
\\

Node A \\
$v_{A} = 12 \si{\volt}$
\\

Node B \\
By KCL, we get the following equation: \\
\begin{align*}
	\frac{v_{B} - 12}{1.6} + \frac{v_{B}}{4} + \frac{v_{B}}{6} &= 0 \\
	v_{C}(\frac{1}{1.6} + \frac{1}{4} + \frac{1}{6}) &= \frac{12}{1.6} \\
	v_{C} &= \frac{12}{1.6}(\frac{1}{1.6} + \frac{1}{4} + \frac{1}{6})^{-1} \\
	v_{C} &= 7.2 \si{\volt}
\end{align*}

\subsection*{Question 1.2.1}

We can reduce the circuit down to a single equivalent resistor and the battery to obtain the current supplied from the battery to the load. The equivalent resistor is given by:

\begin{align*}
R_{eq} &= R_{1.6\si{\ohm}} + R_{4\si{\ohm}}||R_{6\si{\ohm}} \\
&= 4 \si{\ohm}
\end{align*}

\newpage

Now, using Ohm's law, we can solve for the current in the circuit. That is:

\begin{align*}
V &= iR_{eq} \\
i &= \frac{V}{R_{eq}} \\
&= \frac{12}{4} \\
&= 3 \si{\ampere}
\end{align*}

Hence the power supplied from the battery to the load is given by:

\begin{align*}
p &= Vi \\
p &= iR_{eq}i \\
p &= i^{2}R_{eq} \\
&= 3^2 \times 4 \\
&= 36 \si{\watt}
\end{align*}

\subsection*{Question 1.2.2}

This question has been solved using the node voltages found earlier.

The current flowing from Node A to Node B through $R_{1.6 \si{\ohm}}$ is given by:
\begin{align*}
i &= \frac{v_{A} - v_{B}}{R_{1.6 \si{\ohm}}} \\
&= \frac{12 - 7.2}{1.6} \\
&= 3 \si{\ampere}
\end{align*}

The current flowing from Node B to Node C through $R_{4 \si{\ohm}}$ is given by:
\begin{align*}
i &= \frac{v_{B} - v_{C}}{R_{4 \si{\ohm}}} \\
&= \frac{7.2}{4} \\
&= 1.8 \si{\ampere}
\end{align*}

The current flowing from Node B to Node C through $R_{6 \si{\ohm}}$ is given by:
\begin{align*}
i &= \frac{v_{B} - v_{C}}{R_{6 \si{\ohm}}} \\
&= \frac{7.2}{6} \\
&= 1.2 \si{\ampere}
\end{align*}

\subsection*{Question 1.2.2}

This question has been solved using the node voltages found earlier.

The power consumed by resistor $R_{1.6 \si{\ohm}}$ is given by:

\begin{align*}
p &= \frac{(v_{A} - v_{B})^2}{R_{1.6\si{\ohm}}} \\
&= \frac{(12 - 7.2)^2}{1.6} \\
&= 14.4 \si{\watt}
\end{align*}

The power consumed by resistor $R_{4 \si{\ohm}}$ is given by:

\begin{align*}
p &= \frac{(v_{B} - v_{C})^2}{R_{6\si{\ohm}}} \\
&= \frac{(7.2 - 0)^2}{4} \\
&= 12.96 \si{\watt}
\end{align*}

The power consumed by resistor $R_{1.6 \si{\ohm}}$ is given by:

\begin{align*}
p &= \frac{(v_{B} - v_{C})^2}{R_{1.6\si{\ohm}}} \\
&= \frac{(7.2 - 0)^2}{6} \\
&= 8.64 \si{\watt}
\end{align*}

\subsection*{Question 1.2.4}
This question has been solved using the node voltages found earlier.
The voltages across each resistor are as follows:

\begin{align*}
v_{R_{1.6\si{\ohm}}} &= 12 - 7.2 \\
&= 4.8\si{\volt}
\end{align*}

\begin{align*}
v_{R_{4\si{\ohm}}} &= 7.2 - 0 \\
&= 7.2\si{\volt}
\end{align*}

\begin{align*}
v_{R_{6\si{\ohm}}} &= 7.2 - 0 \\
&= 7.2\si{\volt}
\end{align*}

\subsection*{Question 1.2.5}

The aggregate power consumed from the load is equal to 36$\si{\watt}$ which is equal to the power produced by the source.

%----------------------------------------------------------------------------------------
%	PROBLEM 3
%----------------------------------------------------------------------------------------

\section*{Assignment 1.3}

\begin{circuitikz}
	
	\draw (0,0)
	to [sV, l^=$v(t)$] (0,4)
	to [short] (0,4) 
	to [R, l^=40\si{\ohm}] (2,4)
	to [L, l^=127.324\si{\milli\henry}] (4,4)
	to [C, l^=318.3\si{\micro\farad}] (4,0)
	to [short] (0,0)
	;
\end{circuitikz}

The sinusoidal voltage source is 240 volts with a 50 Hz frequency. This means that:
\begin{align*}
f = 50 \si{\hertz} \\
\omega = 2 \pi f = 100 \pi
\end{align*}

Further,

\begin{align*}
V_{rms} &= \frac{V_{m}}{\sqrt{2}} \\
V_{m} &= V_{rms} \times \sqrt{2} \\
&= 240 \sqrt{2}
\end{align*}

Hence,

\begin{align*}
v(t) = 240\sqrt{2} \cos(100\pi t)
\end{align*}

In phasor form,

\begin{align*}
v(t) = 240\sqrt{2}\angle 0\si{\degree}
\end{align*}

\newpage

The impedance of the resistor is $Z_{R} = 40 \si{\ohm}$. The impedance of the inductor is given by:

\begin{align*}
Z_{L} &= j\omega L \\
&= j \times 100\pi \times 127.324e-3 \si{\ohm} \\
&= j40 \si{\ohm} \\
&= 40 \angle 90 \si{\degree} \si{\ohm}
\end{align*}

The impedance of the capacitor is given by:

\begin{align*}
Z_{C} &= \frac{1}{j \omega C} \\
&= \frac{1}{j \times 100\pi \times 318.3e-6} \si{\ohm} \\
&= \frac{10}{j} \si{\ohm} \\
&= -j10 \si{\ohm} \\
&= 10 \angle -90 \si{\degree} \si{\ohm} \\
\end{align*}

\subsection*{Question 1.3.1}
Now to find the current we can sum the impedances (because they are in series) and apply Ohm's law to solve for the current.

\begin{align*}
Z_{eq} &= Z_{R} + Z_{L} + Z_{C} \\
&= 40 + j40 - j10 \\
&= 40 + j30 \\
&= 50 \angle 36.86 \si{\degree}
\end{align*}

Hence, the current is given by:
\begin{align*}
\textbf{V} &= \textbf{I}Z_{eq} \\
\textbf{I} &= \frac{\textbf{V}}{Z_{eq}} \\
&= \frac{240 \sqrt{2} \angle 0 \si{\degree}}{50 \angle 36.86 \si{\degree}} \\
&= 6.79 \angle -36.86\si{\degree}
\end{align*}

Hence,
\begin{align*}
I_{rms} &= \frac{6.79}{\sqrt{2}} \\
&= 4.80 \si{\ampere}
\end{align*}

Now, we get the real power according to the following formula:

\begin{align*}
P = V_{rms}I_{rms}\cos(\theta),
\end{align*}

where $\theta$ is the difference between the voltage and current phase angles. \\

Hence,

\begin{align*}
P &= 240 \times 4.8 \times \cos(-36.86) \\
&= 921 \si{\watt}
\end{align*}

The average power consumed by the resistor is given by:

\begin{align*}
P &= I_{rms}^2 R \\
&= 4.8^2 \times 40 \\
&= 921.60 \si{\watt}
\end{align*}

The two values align, indicating that the resistor consumes all of the real power delivered by the source.

\subsection*{Question 1.3.2}

The power factor at which the power is delivered is given by:

\begin{align*}
PF &= \cos(\theta)
&= \cos(-36.86\si{\degree})
&= 0.8
\end{align*}

\subsection*{Question 1.3.3}

%The current is leading the voltage, since the power angle is -36.86 $\degree$. This means that the alternating current reaches its maximum value before the alternating voltage.

\subsection*{Question 1.3.4}

The voltage across the resistor is given by:

\begin{align*}
\textbf{V}_{R} &= \textbf{I}R \\
&= 6.79 \angle -36.86\si{\degree} \times 40 \angle 0\si{\degree}\\
&= 271.6 \si{\degree} -36.86\degree \si{\volt}
\end{align*}

\newpage

The voltage across the inductor is given by:

\begin{align*}
\textbf{V}_{L} &= \textbf{I}Z_{L} \\
&= 6.79 \angle -36.86\degree \times 40 \angle 90\degree \\
&= 271.6 \angle 53.14\degree \volt
\end{align*}

The voltage across the capacitor is given by:

\begin{align*}
\textbf{V}_{C} &= \textbf{I}Z_{C} \\
&= 6.79 \angle -36.86\degree \times 10 \angle -90\degree \\
&= 67.9 \angle -126.86\degree \volt
\end{align*}



\end{document}